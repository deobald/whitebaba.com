\documentclass{article}

\usepackage{graphicx}
\usepackage{subcaption}
\usepackage{float}
\usepackage{hyperref}

\setlength{\parindent}{1.3em}
\setlength{\parskip}{0.7em}

\begin{document}

\begin{titlepage}
   \vspace*{\stretch{1.0}}
   \begin{center}
     \Huge\textbf{Real Work}\\
     \vspace{5cm}
     \large\textit{Steven Deobald}\\
     \large\textit{Version 1.0}\\
     \large\textit\today
   \end{center}
   \vspace*{\stretch{2.0}}
\end{titlepage}

``Real Work'' is the product of real labour. Real labour is the labours which are felt viscerally by the labourer. I ostensibly write software for a living. Or manage a business. Or sell peoples' time. Something like this. Regardless of the specific activity, none of these describes ``real work''... at least not on most days.

It is possible for the activity of writing softare to feel like Real Work. Once any activity feels like Real Work it becomes Real Work. On these days, one feels accomplished. One feels like the labours of the day were worthwhile, were whole, were satisfying. It is hard to find this satisfaction in my career: in software, in business, in sales. People in these positions will sometimes fabricate meaning in their work to make themselves feel more important. We wear formal business attire, we write mottos and mission statements for ourselves, and we try to build castles out of abstract ideas which we hope will penetrate the Earth somehow to reify themselves as Something Real. Sometimes they do. But not often.

The work which is Often Real --- the work which is Always Real --- is the Real Work of the title. This work is inherently valuable. Construction of a new building, the cleaning of the floors and vessels, the education of children, the healing of a sick person, the cutting of hair, the repair of broken cookware, the growing of necessary plants. The list goes on.

For five years in Bangalore I have not had a maid. I have prided myself on not having a maid and although my apartment would occasionally suffer from dust collected because I was incapable of staying abreast of its daily deposit from nearby construction sites and diesel engines, I knew in my heart that if I could keep my little Bangalore apartment clean, I could keep my house clean in other parts of the Earth with little worry.

After five years, I have moved into a larger home. One big enough for two people and guests. This home is too much space for me to clean and so we have hired a maid. She is a wonderful woman with a wonderful sister who accompanies her to come clean our new home. And yet, when she arrives, I am plagued with guilt and I often leave the house. Why?

More often than not, when she arrives, I am ``working'' at my computer. Her Work puts my Work to shame. What do I accomplish in the time she cleans this entire house? Perhaps I write one email. Perhaps I modify one document. At best. Such work is not inherently valuable and often I will find, some days later, that the work was in fact counterproductive or worthless. Her work is never worthless. Without her, a dozen homes in my neighbourhood would fall into disrepair.

It is strange that much of my time is spent searching for Real Work when it is right under my nose.

May the world bless this work of inherent value and may it always be valued as it is performed.

\end{document}
